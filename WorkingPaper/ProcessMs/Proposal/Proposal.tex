%% Template for a preprint Letter or Article for submission
%% to the journal Nature.
%% Written by Peter Czoschke, 26 February 2004
%%
\documentclass[12pt]{article}
%\bibliographystyle{naturemag}
\usepackage[utf8]{inputenc}
\usepackage[T1]{fontenc}
\usepackage{amsmath}
\usepackage[mathlines,displaymath]{lineno}
\usepackage{mathtools}
% amssymb package, useful for mathematical symbols
\usepackage{amssymb}
%Proposals should be no more than 300 words long, describe the nature
%and novelty of the work, the contribution of the proposed article to
%the discipline, and the qualifications of the author(s) who will write
%the manuscript

%Ideas and Perspectives. Ecology Letters is particularly interested in
%novel essays expressing new ideas and perspectives that will appeal to
%a wide ecological audience. It is important that Ideas and
%Perspectives be focused on a topic of current interest. We are
%interested in new ideas, emerging frameworks, and controversial
%perspectives on hot areas of research. There is a need to present a
%view that is sufficiently complete to convince reviewers of the value
%of the contribution. There is the same expectation for the novelty of
%Ideas and Perspectives as for Letters. Those articles principally
%reviewing a topic, those that are just as statement of opinion, and
%those primarily discussing the author's own work will not be
%considered. Articles that are successful usually present a
%quantitative analysis, as a way of introducing a new perspective in
%ecology. Authors interested in submitting such a manuscript should
%first send a one-paragraph proposal (no more than 300 words) to the
%Editorial Office (see above).

\begin{document}

%Dear Editors,
\\
%Please find attached the proposal of our ms ``Metacommunities in dynamic
%landscapes'' to be submitted to ``Ideas and Perspectives'' in {\em
%  Ecology Letters}.

Predictions from experiments and field data have shown that high landscape connectivity promotes higher species richness than low connectivity but this pattern is not general because examples demonstrating high diversity in low connected landscapes also exist. Landscape connectivity shows intraday and daily to seasonal or larger time scale fluctuations but their role to understand the effect of landscape connectivity on species richness is lacking in metacommunity theory. Here, we connect the many factors that drive landscape connectivity by varying the amplitude and frequency determining the dispersal radius in spatial networks to show that landscape connectivity play a key role in predicting species richness. 

We show that the fluctuations of landscape connectivity support metacommunities with higher species richness than static landscapes in fragmented landscapes. We show that the variance of regional species richness peaks with an intermediate number of isolated components in dynamic landscapes suggesting high regional species richness can also occur in dynamic landscapes with highly fragmented landscapes. Our results also show a dispersal radius threshold below which species richness drops dramatically in static landscapes. This threshold is not observed in dynamic landscapes for a broad range of amplitude and frequency values determining landscape connectivity.

Our analysis suggest that landscape connectivity together with patch dynamics into metacommunity theory can provide new testable predictions about species diversity at local and regional scales in fast-changing landscapes. Our work will be of interest to theoreticians and experimentalists working at the interface of biodiversity and landscape dynamics at short and large spatiotemporal scales.

Jan Klecka and Gian Marco Palamara have extensive experience in combining experiments and modeling in ecological systems. Charles de Santana and Carlos J. Meli\'an have extensive experience in modeling biodiversity dynamics and food webs.

We thank you in advance for the attention dedicated to our manuscript

Yours sincerely,
\vspace{0.1 in}

\noindent Charles de Santana (corresponding author: charles.desantana@eawag.ch)\\
Jan Klecka\\
Gian Marco Palamara\\
Carlos J. Meli\'an\\
\end{document}


