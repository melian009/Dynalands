\documentclass{article}
\usepackage[utf8]{inputenc}

\pagenumbering{gobble}
\begin{document}
\parskip 6pt
\baselineskip 12pt
\noindent Dear Dr. Holyoak, dear members of the Editorial Board,\\
\\

Please find attached our manuscript entitled \emph{``Metacommunities in dynamic landscapes``} to be considered for publication as a \emph{Letter} in \textbf{Ecology Letters}. In this manuscript we explore the effect of fluctuations in landscape connectivity on species richness. 

Predicting biodiversity dynamics in rapidly changing and human modified landscapes is one of the most important problems in ecology and society because most species extinctions in this period are driven by habitat loss and landscape fragmentation. Several studies dealing with habitat loss and landscape fragmentation have shown thresholds of habitat destruction beyond which there is a rapid avalanche of species extinctions. Apart from such directional human-driven changes, fluctuations in landscape connectivity show intraday and daily to seasonal or larger time scale fluctuations but the consequences of fluctuations in landscape connectivity for species richness received less attention in metacommunity theory.

The interplay between local patch dynamics and larger scale spatial processes is well documented in driving species richness and coexistence in metacommunities. However, landscape dynamics encompasses two major processes, i.e., patch dynamics and variation in landscape connectivity. Predictions from experiments and field data have shown that high landscape connectivity promotes higher species richness than low connectivity but this pattern is not general because examples demonstrating high diversity in low connected landscapes also exist. 

Here, we connect the many factors that drive landscape connectivity to species richness by varying the amplitude and frequency determining the dispersal radius in spatial networks. We show that the fluctuations of landscape connectivity support metacommunities with higher species richness than static landscapes. We found that the variance of regional species richness peaks with an intermediate degree of fragmentation in dynamic landscapes suggesting high regional species richness can also occur in dynamic landscapes with highly fragmented landscapes. Our results also show a dispersal radius threshold below which species richness drops dramatically in static landscapes. This threshold is not observed in dynamic landscapes for a broad range of amplitude and frequency values determining landscape connectivity. Our analysis suggest that landscape connectivity together with patch dynamics can provide new testable predictions about species diversity at local and regional scales in rapidly changing landscapes.

We believe our work represents an important advance to understand metacommunity dynamics in fragmented and rapidly changing landscapes. Our results suggest that changes in biodiversity inferred only from static landscapes, from landscapes with only patch dynamics or from landscapes with only directional habitat loss and fragmentation are likely to be improved taking into account fluctuations in landscape connectivity. Our work will be of interest to experimentalists, managers and theoreticians working at the interface of biodiversity, conservation and landscape dynamics at short and large spatiotemporal scales. Our manuscript contains a novel model taking into account the effect of fluctuations in landscape connectivity to predict regional species richness. This work is outstandingly novel relative to the work published by one of the authors and cited in this ms (uploaded with the submission). 

We suggest the following potential editors and reviewers.
Potential editors are:

\begin{itemize}
\item Marcel Holyoak
\item Sergio Navarrete
\end{itemize}

Possible reviewers are:

\begin{itemize}
\item Otso Ovaskainen
\item Juan E. Keymer
\item Joel Rybicki
\end{itemize}

All authors have agreed to the submission of the manuscript and all persons entitled to authorship have been named. The work has not been submitted to any other journal and it represents our original research. We would like to mention that we have sent a proposal for the submission of this manuscript to \emph{Ideas and Perspectives} section. The proposal was rejected. However, the Editor-in-Chief Prof. Marcel Holyoak has encouraged us to submit this ms. as a Letter.

We thank you in advance for the attention dedicated to our manuscript

Yours sincerely,
\vspace{0.1 in}

\noindent Charles de Santana (corresponding author: charles.desantana@eawag.ch)\\
Jan Klecka\\
Gian Marco Palamara\\
Carlos J. Meli\'an\\
\end{document}

